\documentclass[12pt]{article}

\usepackage[english]{babel}
\usepackage[utf8x]{inputenc}
\linespread{1}
\pagestyle{empty}
\setlength\parindent{12pt}


\usepackage[top=1in,bottom=1in,left=1in,right=1in]{geometry}
\title{6.852 Distributed Algorithms Project Proposal}
\author{Ariel Anders and Noel Hollingsworth}

\begin{document}
\begin{center}{\bf \Large
6.852 Distributed Algorithms Project Proposal\\ }
Ariel Anders and Noel Hollingsworth 
\end{center}
We propose to do a reading project on intersection collision avoidance by decentralized autonomous intersection management.  \\

Traffic congestion is one of the leading causes of lost productivity and decreased standard of living in urban settings. Recent advances in artificial intelligence suggest vehicle navigation by autonomous agents will be possible in the near future.\cite{dresner}  We plan to investigate approaches for alleviating traffic congestion, specifically at intersections.  There have been advances in developing centralized controls to pilot cars at intersections; however, in light of the course we plan to focus our literary review on decentralized intersection management methods.

In particular, we are interested in a line of research by \cite{naumann} which creates a
token-ring-based communication protocol among autonomous vehicles in a specified zone around the
intersection.  Collision avoidance is ensured by means of semaphor-based algorithms, which allow
only one vehicle to remain in the intersection segment at a time.  In our literary review, we would
like to show the connection between decentralized intersection management methods and the mutual
exclusion algorithms we reviewed in class.  One aim of our project is to translate this problem into
the current distributed algorithms literature, for example, we would like to discuss the liveness
properties and show the fairness properties of the algorithms as in \cite{tszchiu} for
centralized intersection management. Due to the nature of
this problem, we will need to include a discussion of modeling the problem, which may take into
account the individual car's current position, velocity, and acceleration while approaching the
intersection. 
\begin{thebibliography}{9}

\bibitem{dresner}
Kurt Dresner and Peter Stone. ''Multiagent Traffic Management: An Improved Intersection
Control Mechanism", AAMAS'05 Proceedings of the fourth international joint conference on Autonomous agents and multiagent systems, New York, NY, USA, 2005.
\bibitem{naumann}
Naumann, Rolf, and Rainer Rasche. ``Intersection collision avoidance by means of decentralized security and communication management of autonomous vehicles", Univ.-GH, SFB 376, 1997.
\bibitem{tszchiu}
Au, Tsz-Chiu, Neda Shahidi, and Peter Stone. ``Enforcing Liveness in Autonomous Traffic Management." AAAI. 2011.
\end{thebibliography}
\end{document}
